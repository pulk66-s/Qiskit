\documentclass{article}
\usepackage{amsmath}
\begin{document}
\title{Gram-Schmidt method}
The Gram-Schmidt method is a method to transform a set of vectors into an orthogonal set of vectors.

\section{Definition}

Let $V$ be a vector space with an inner product $\langle\cdot,\cdot\rangle$ and $S = \{v_1, v_2, \dots, v_n\}$ a set of vectors of $V$.
By definition the projection of $V$ on $U$ is:
\begin{equation}
    P_u(v) = \frac{v \cdot u}{u \cdot u}u
\end{equation}
The Gram-Schmidt method is defined as:
\begin{equation}
    \begin{split}
        w_1 & = v_1 \\
        w_2 & = v_2 - P_{w_1}(v_2) \\
        w_3 & = v_3 - P_{w_1}(v_3) - P_{w_2}(v_3) \\
        & \vdots \\
        w_n & = v_n - P_{w_1}(v_n) - P_{w_2}(v_n) - \dots - P_{w_{n-1}}(v_n)
    \end{split}
\end{equation}

The process can be summarized as:
\begin{equation}
    w_n = v_n - \sum_{i=1}^{n-1} P_{w_i}(v_n)
\end{equation}

\section{Proof}

\subsection{Two vector orthogonal}

Two vector $A$ and $B$ are orthogonal if $A \cdot B = 0$.

\begin{equation}
    \begin{split}
        A & = \begin{pmatrix}x_1 \\ x_2 \\ \dots \\ x_n\end{pmatrix},
            B = \begin{pmatrix} y_1 \\ y_2 \\ \dots \\ y_n\end{pmatrix} \\
        A \cdot B & = \sum_{i=1}^n x_iy_i \\
    \end{split}
\end{equation}

If the two vectors are orthogonal then $\sum_{i=1}^{n} x_iy_i = 0$.

\subsection{Gram-Schmidt vectors}

\begin{equation}
    \begin{split}
        w_n & = v_n - \sum_{i=1}^{n-1} P_{w_i}(v_n) \\
        w_{n+1} & = v_{n+1} - \sum_{i=1}^{n} P_{w_i}(v_{n+1}) \\
        \Longrightarrow w_n \cdot w_{n+1} & = \left(v_n - \sum_{i=1}^{n-1} P_{w_i}(v_n)\right) \cdot \left(v_{n+1} - \sum_{i=1}^{n} P_{w_i}(v_{n+1})\right) \\
        & = v_n \cdot v_{n+1} - \sum_{i=1}^{n-1} v_n \cdot P_{w_i}(v_{n+1}) - \sum_{i=1}^{n} P_{w_i}(v_n) \cdot v_{n+1} + \sum_{i=1}^{n-1} \sum_{j=1}^{n} P_{w_i}(v_n) \cdot P_{w_j}(v_{n+1}) \\
        
    \end{split}
\end{equation}

\end{document}
