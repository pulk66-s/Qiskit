\documentclass{article}
\usepackage{amsmath}
\begin{document}
\title{Inner product}

\section{Definition}
Notation:
\begin{equation}
    (|\psi\rangle,|\phi\rangle)
\end{equation}
Where $|\psi\rangle$ and $|\phi\rangle$ are vectors in a Vector space.
in $C^n$:
\begin{equation}
    \begin{split}
        |\psi\rangle & = \begin{pmatrix} \psi_1 \\ \psi_2 \\ \vdots \\ \psi_n \end{pmatrix} \\
        |\phi\rangle & = \begin{pmatrix} \phi_1 \\ \phi_2 \\ \vdots \\ \phi_n \end{pmatrix} \\
        \Longrightarrow (|\psi\rangle,|\phi\rangle) & = \sum_{i=1}^{n}\psi_i^*\phi_i
    \end{split}
\end{equation}

\section{Properties}

\subsubsection{Right linearity}
\begin{equation}
    (|\psi\rangle,\alpha|\phi\rangle) = \alpha(|\psi\rangle,|\phi\rangle)
\end{equation}

Proof:
\begin{equation}
    \begin{split}
        (|\psi\rangle,\alpha|\phi\rangle) & = \sum_{i=1}^{n}\psi_i^*\alpha\phi_i \\
        & = \alpha\sum_{i=1}^{n}\psi_i^*\phi_i \\
        & = \alpha(|\psi\rangle,|\phi\rangle) \\
        \Longrightarrow (|\psi\rangle,\alpha|\phi\rangle) & = \alpha(|\psi\rangle,|\phi\rangle) \\
    \end{split}
\end{equation}

\subsubsection{Conjugate symmetry}
\begin{equation}
    (|\psi\rangle,|\phi\rangle) = (|\phi\rangle,|\psi\rangle)^*
\end{equation}

Proof:
\begin{eqnarray}
    \begin{split}
        (|\psi\rangle,|\phi\rangle) & = \sum_{i}\psi_i^*\phi_i \\
        & = \sum_{i}\phi_i^*\psi_i, \\
        & = (|\phi\rangle,|\psi\rangle)^* \\
        \Longrightarrow (|\psi\rangle,|\phi\rangle) & = (|\phi\rangle,|\psi\rangle)^* \\
    \end{split}
\end{eqnarray}

Explanation why $\psi^*\phi = \phi^*\psi$:
\begin{equation}
    \begin{split}
        \psi & = \alpha + i\beta, \phi = \gamma + i\delta \\
        \psi^*\phi & = (\alpha - i\beta)(\gamma + i\delta) \\
        & = \alpha\gamma + \beta\delta + \alpha\delta i - \beta\gamma i \\
        & = (\gamma - i\delta)(\alpha + i\beta) \\
        & = \phi^*\psi \\
        \Longrightarrow \psi^*\phi & = \phi^*\psi \\
    \end{split}
\end{equation}

\subsubsection{Inequality}
\begin{equation}
    (|\psi\rangle,|\psi\rangle) \geq 0, \iff |\psi\rangle = 0
\end{equation}

Proof:
\begin{equation}
    \begin{split}
        |\psi\rangle & = 0 = \begin{pmatrix} 0 \\ 0 \\ \vdots \\ 0 \end{pmatrix} \\
        (|\psi\rangle,|\psi\rangle) & = \sum_{i=1}^{n}\psi_i^*\psi_i \\
        & = \sum_{i=1}^{n}0 \\
        & = 0 \\
        \Longrightarrow (|\psi\rangle,|\psi\rangle) & \geq 0 \\
    \end{split}
\end{equation}

\subsubsection{Left conjugate linearity}

\begin{equation}
    (\alpha|\psi\rangle,|\phi\rangle) = \alpha^*(|\psi\rangle,|\phi\rangle)
\end{equation}

Proof:
\begin{equation}
    \begin{split}
        (\alpha|\psi\rangle,|\phi\rangle) & = \sum_{i=1}^{n}\alpha\psi_i^*\phi_i \\
        & = \alpha^*\sum_{i=1}^{n}\psi_i^*\phi_i \\
        & = \alpha^*(|\psi\rangle,|\phi\rangle) \\
        \Longrightarrow (\alpha|\psi\rangle,|\phi\rangle) & = \alpha^*(|\psi\rangle,|\phi\rangle) \\
    \end{split}
\end{equation}

\end{document}
