\documentclass{article}
\usepackage{qcircuit}
\usepackage{amsmath}
\begin{document}
\title{Oracle Gate}

\section{Introduction}
This gate is used to apply a function $f(x)$ on a qubit.

\section{Circuit}
\Qcircuit {
    |x\rangle & \qw & \multigate{1}{U_f} & \qw & |x\rangle \\
    |y\rangle & \qw & \ghost{U_f} & \qw & |y \oplus f(x)\rangle
}

Arithmetic representation:
\begin{equation}
    \begin{split}
        |\psi_0\rangle & = |x\rangle |y\rangle \\
        |\psi_1\rangle & = |x\rangle |y \oplus f(x)\rangle \\
    \end{split}
\end{equation}

\section{Special Cases}
\subsection{y = 0}
\begin{equation}
    \begin{split}
        |\psi_0\rangle & = |x\rangle |0\rangle \\
        \Longrightarrow |\psi_1\rangle & = |x\rangle |f(x)\rangle \\
    \end{split}
\end{equation}

\section{Development}
\subsection{Development and simplification}

\begin{equation}
    \begin{split}
        |\psi_0\rangle & = |x\rangle |y\rangle \\
        |x\rangle & = \alpha |0\rangle + \beta |1\rangle \\
        |y\rangle & = \gamma |0\rangle + \delta |1\rangle \\
        |\psi_1\rangle & = |x\rangle |y \oplus f(x)\rangle \\
        \Longrightarrow |\psi_1\rangle & = (\alpha |0\rangle + \beta |1\rangle)|(\gamma |0\rangle + \delta |1\rangle) \oplus f(x)\rangle \\
    \end{split}
\end{equation}

\subsection{Example}
\begin{equation}
    \begin{split}
        |\psi_0\rangle & = |x\rangle|y\rangle \\
        |x\rangle & = \frac{|0\rangle + |1\rangle}{\sqrt{2}} + \frac{1}{\sqrt{2}}|0\rangle + \frac{1}{\sqrt{2}}|1\rangle \\
        |y\rangle & = 1|0\rangle + 0|1\rangle \\
        |\psi_1\rangle & = (\alpha |0\rangle + \beta |1\rangle)|(\gamma |0\rangle + \delta |1\rangle) \oplus f(x)\rangle \\
        \Longrightarrow |\psi_1\rangle & = (\frac{1}{\sqrt{2}}|0\rangle + \frac{1}{\sqrt{2}}|1\rangle)|0 \oplus f(x)\rangle \\
        & = \frac{|0\rangle + |1\rangle}{\sqrt{2}}|f(x)\rangle \\
        & = \frac{|0, f(x)\rangle + |1, f(x)\rangle}{\sqrt{2}} \\
    \end{split}
\end{equation}

\end{document}