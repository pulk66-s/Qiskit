\documentclass{article}
\usepackage{qcircuit}
\usepackage{amsmath}
\begin{document}
\title{Deutsch Algorithm}
\section{Introduction}
This document explain the behavior of the Deutsch Algorithm, this algorithm is a proof
of the Quantum Supremacy, it is a single algorithm that can be solved in a quantum computer
and can prove that the quantum computer is faster than a classical computer. By 
solving a classical algorithm that has a time complexity of $O(2^n)$ in a quantum computer
that has a time complexity of $O(1)$.

\section{Problem}

The problem that has been used in the Deutsch Algorithm is the Constant/Equilibrium problem 
of a function. The problem is defined as follows:
Take a function $f:\{0,1\}^n \rightarrow \{0,1\}$, and we want to know if the function is
constant or equilibrium. A constant function is a function that returns the same value
for all the inputs, and an equilibrium function is a function that returns $0$ for half
of the inputs and $1$ for the other half.

\section{Oracle Circuit}
This is the circuit for two qubits $|x\rangle$ and $|y\rangle$:

\Qcircuit {
    |x\rangle & \multigate{1}{U_f} & \qw & |x\rangle \\
    |y\rangle & \ghost{U_f} & \qw & |y \oplus f(x)\rangle
}

Where $|y \otimes f(x)\rangle$ is the modulo 2 sum of $y$ and $f(x)$. This circuit
is called an oracle.

If we set $y$ to $|0\rangle$ and $x$ to $|0\rangle$ we will have the following circuit:

\Qcircuit {
    |x\rangle & \multigate{1}{U_f} & \qw & |x\rangle \\
    |0\rangle & \ghost{U_f} & \qw & |0 \oplus f(x)\rangle = |f(x)\rangle
}

Instead of the $|0\rangle$ we can use the $|-\rangle$ value that equals $\frac{1}{\sqrt{2}}(|0\rangle - |1\rangle)$:

\Qcircuit {
    |x\rangle & \multigate{1}{U_f} & \qw & (-1)^{f(x)}|x\rangle \\
    |-\rangle & \ghost{U_f} & \qw & |-\rangle
}

The proof is in section 4

\section{Proof of the oracle circuit with 0 and -}

\begin{equation}
    \begin{split}
        |\psi_0\rangle & = |x\rangle|-\rangle \\
        |-\rangle & = \frac{1}{\sqrt{2}}(|0\rangle - |1\rangle) \\
        \Longrightarrow |\psi_0\rangle & = \frac{1}{\sqrt{2}}(|x\rangle|0\rangle - |x\rangle|1\rangle) \\
        |\psi_1\rangle & = \frac{1}{\sqrt{2}}(U_f(|x\rangle|0\rangle) - U_f(|x\rangle|1\rangle)) \\
        |\psi_1\rangle & = \frac{1}{\sqrt{2}}(|x\rangle|f(x)\rangle - |x\rangle|1 \oplus f(x)\rangle) \\
    \end{split}
\end{equation}

f is a function that returns 0 or 1 so $1 \oplus f(x)$ is the invert of $f(x)$ that is noted $\overline{f(x)}$

\begin{equation}
    \Longrightarrow |\psi_1\rangle = \frac{1}{\sqrt{2}}(|x\rangle|f(x)\rangle - |x\rangle|\overline{f(x)}\rangle)
\end{equation}

\begin{equation}
    \begin{split}
        & \begin{cases}
            \text{If $f(x)$ = 0:} & |\psi_1\rangle = \frac{1}{\sqrt{2}}(|x\rangle|0\rangle - |x\rangle|1\rangle)
            = |x\rangle|-\rangle \\
            \text{If $f(x)$ = 1:} & |\psi_1\rangle = \frac{1}{\sqrt{2}}(- |x\rangle|0\rangle + |x\rangle|1\rangle)
            = -|x\rangle|-\rangle\\
        \end{cases} \\
        \Longrightarrow |\psi_1\rangle & = (-1)^{f(x)}|x\rangle|-\rangle \\
        \text{Because $(-1)^0 = 1$ and $(-1)^1 = -1$} \\
    \end{split}
\end{equation}

\section{Deutsch Circuit}

The actual Deutsch circuit is the following:

\Qcircuit {
    |0\rangle & \qw & \gate{H} & \multigate{1}{U_f} & \gate{H} & \meter \\
    |0\rangle & \gate{X} & \gate{H} & \ghost{U_f} & \qw & \qw
}

Let's calculate the output of the circuit:

\begin{equation}
    \begin{split}
        \text{Beginning state:} & \\
        \Longrightarrow |\psi_0\rangle & = |0\rangle|0\rangle
    \end{split}
\end{equation}

Applying the first $X$ gate of the second qubit:
\begin{equation}
    \Longrightarrow |\psi_1\rangle = |0\rangle X|0\rangle = |0\rangle|1\rangle
\end{equation}

Applying the $H$ gate of the two qubits:

\begin{equation}
    \begin{split}
        |\psi_2\rangle & = H|0\rangle H|1\rangle = \frac{1}{\sqrt{2}}(|0\rangle + |1\rangle)\frac{1}{\sqrt{2}}(|0\rangle - |1\rangle) \\
        \Longrightarrow |\psi_2\rangle & = |+\rangle|-\rangle \\
    \end{split}
\end{equation}

Applying $f(x)$ to all the qubits:

\begin{equation}
    \begin{split}
        |\psi_3\rangle & = U_f(|+\rangle|-\rangle) \\
        |\psi_3\rangle & = U_f(\frac{1}{\sqrt{2}}(|0\rangle + |1\rangle)|-\rangle) \\
        |\psi_3\rangle & = U_f(\frac{1}{\sqrt{2}}(|0\rangle|-\rangle + |1\rangle|-\rangle)) \\
        |\psi_3\rangle & = \frac{1}{\sqrt{2}}(U_f|0\rangle|-\rangle + U_f|1\rangle|-\rangle) \\
        |\psi_3\rangle & = \frac{1}{\sqrt{2}}((-1)^{f(0)}|0\rangle|-\rangle + (-1)^{f(1)}|1\rangle|-\rangle) \\
        |\psi_3\rangle & = \frac{1}{\sqrt{2}}((-1)^{f(0)}|0\rangle + (-1)^{f(1)}|1\rangle)|-\rangle \\
        & \text{We can delete $|-\rangle$ because it will not being used anymore and we will not be mesured}: \\
        \Longrightarrow |\psi_3\rangle & = \frac{1}{\sqrt{2}}((-1)^{f(0)}|0\rangle + (-1)^{f(1)}|1\rangle)
    \end{split}
\end{equation}

If $f(x)$ is constant then $f(0) = f(1)$ and $(-1)^{f(0)} = (-1)^{f(1)}$ so the state will be:
\begin{equation}
    \begin{cases}
        \text{If $f(x) = 0$:} & |\psi_3\rangle = \frac{1}{\sqrt{2}}(|0\rangle + |1\rangle) = |+\rangle \\
        \text{If $f(x) = 1$:} & |\psi_3\rangle = \frac{1}{\sqrt{2}}(-|0\rangle - |1\rangle) = -|+\rangle \\
    \end{cases}
\end{equation}

Else:
\begin{equation}
    \begin{cases}
        \text{If $f(0) = 0$ and $f(1) = 1$:} & |\psi_3\rangle = \frac{1}{\sqrt{2}}(|0\rangle - |1\rangle) = |-\rangle \\
        \text{If $f(0) = 1$ and $f(1) = 0$:} & |\psi_3\rangle = \frac{1}{\sqrt{2}}(-|0\rangle + |1\rangle) = -|-\rangle \\
    \end{cases}
\end{equation}

Finally We can say that:
\begin{equation}
    \begin{cases}
        \text{If $f(x)$ is constant: } & |\psi_3\rangle = \pm|+\rangle \\
        \text{If $f(x)$ is equilibrium: } & |\psi_3\rangle = \pm|-\rangle \\
    \end{cases}
\end{equation}

Applying $H$ to the first qubit:
\begin{equation}
    \begin{split}
        & H|-\rangle = \frac{1}{\sqrt{2}}(H|0\rangle - H|1\rangle) \\
        & = \frac{1}{\sqrt{2}}(\frac{1}{\sqrt{2}}(|0\rangle + |1\rangle) - \frac{1}{\sqrt{2}}(|0\rangle - |1\rangle)) \\
        H|-\rangle & = |1\rangle \\
        & H|+\rangle = \frac{1}{\sqrt{2}}(H|0\rangle + H|1\rangle) \\
        & = \frac{1}{\sqrt{2}}(\frac{1}{\sqrt{2}}(|0\rangle + |1\rangle) + \frac{1}{\sqrt{2}}(|0\rangle - |1\rangle)) \\
        & H|+\rangle = |0\rangle \\
        & \begin{cases}
            \text{If $f(x)$ is constant: } & |\psi_4\rangle = H|\psi_3\rangle = \pm H|-\rangle \\
            \text{If $f(x)$ is equilibrium: } & |\psi_4\rangle = H|\psi_3\rangle = \pm H|+\rangle \\
        \end{cases} \\
        \Longrightarrow & \begin{cases}
            \text{If $f(x)$ is constant: } & |\psi_4\rangle = \pm|1\rangle \\
            \text{If $f(x)$ is equilibrium: } & |\psi_4\rangle = \pm|0\rangle \\
        \end{cases} \\
    \end{split}
\end{equation}

\section{Conclusion}

We can see that if $f(x)$ is constant then the first qubit will be $|1\rangle$ and if $f(x)$ is equilibrium
then the first qubit will be $|0\rangle$. So if we measure the first qubit we will know if $f(x)$ is constant
or equilibrium. This is the proof of the Deutsch Algorithm and the Quantum Supremacy.

\end{document}