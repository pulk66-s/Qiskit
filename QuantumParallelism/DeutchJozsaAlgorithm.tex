\documentclass{article}
\usepackage{amsmath}
\usepackage{qcircuit}
\begin{document}
\title{Explanation of the Quantum Parallelism with Deutch-Jozsa}
\section{Introduction}
The Quantum parallelism is a way to apply a function at different states of qbits
in the same time, so now we can transform a complexity of $O(n^k)$ to $O(1)$.
This complexity is the number of mesures of the function.

The Deutch-Jozsa algorithm is a more general version of the Deutch algorithm.

\section{Circuit}
\Qcircuit {
    |0\rangle & \gate{H^{\otimes n}} & \multigate{1}{U_f} & \gate{H^{\otimes n}} & \meter \\
    |1\rangle & \gate{H} & \ghost{U_f} & \qw & \qw
}

This circuit has 4 main states:
\begin{equation}
    \begin{cases}
        |\psi_0\rangle & \text{Is the beginning state} \\
        |\psi_1\rangle & \text{Is the state after the first layer of H gates} \\
        |\psi_2\rangle & \text{Is the state after the oracle U gate} \\
        |\psi_3\rangle & \text{Is the state after the second layer of H gates} \\
    \end{cases}
\end{equation}

\section{Calculation}

Starting state
\begin{equation}
    |\psi_0\rangle = |0\rangle^{\otimes n}|1\rangle
\end{equation}

Applying the first layer of H gates
\begin{equation}
    \begin{split}
        |\psi_1\rangle & = H|\psi_0\rangle \\
        & = H|0\rangle^{\otimes n}H|1\rangle \\
        H|0\rangle & = \frac{1}{\sqrt{2}}(|0\rangle + |1\rangle) = |+\rangle \\
        H|1\rangle & = \frac{1}{\sqrt{2}}(|0\rangle - |1\rangle) = |-\rangle \\
        \Longrightarrow |\psi_1\rangle & = \sum_{i=1}^{i=n}(|+\rangle)|-\rangle \\
    \end{split}
\end{equation}

\end{document}