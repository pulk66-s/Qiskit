\documentclass{article}
\usepackage{qcircuit}
\usepackage{amsmath}
\begin{document}
\section{Introduction}
This is a simple example of quantum teleportation.
The quantum teleportation works using a bell state defined as:
\begin{equation}
    |\Phi^+\rangle = \frac{1}{\sqrt{2}}(|00\rangle + |11\rangle)
\end{equation}
The goal of the quantum teleportation is to share this 2-qubit state between 2 persons
Traditionnally called Alice and Bob. So Alice has the first qubit and Bob the second one.
The quantum teleportation will shows that Alice can send her qubit state to Bob without
sending the qubit itself. It proves that the quantum information can be transfered without
any classical information and at an infinite speed.

When Alice mesure her qubit, she will automatically change the state of Bob's qubit
to the same state as her qubit. This is the quantum teleportation.

\section{Circuit}
This is the circuit of the quantum teleportation:

\Qcircuit {
    & |\psi\rangle & \ctrl{1} & \gate{H} & \meter & \cw \\
    & |\beta_0\rangle & \gate{X} & \meter & \cw  \\
    & |\beta_1\rangle & \qw & \qw & \gate{X} \cwx[-1] & \gate{Z} \cwx[-2] \\
}

Where $|\beta\rangle = |\beta_0\beta_1\rangle = \frac{1}{\sqrt{2}}|00\rangle + \frac{1}{\sqrt{2}}|11\rangle$
A Bell State, $|\beta_0\rangle$ is the first qubit and $|\beta_1\rangle$ the second one.

At this state $|\beta_0\rangle$ and $|\beta_1\rangle$ are entangled. So if we mesure $|\beta_0\rangle$,
$|\beta_1\rangle$ will be in the same state as $|\beta_0\rangle$.

We are applying a CNOT gate on $|\psi\rangle$ and $|\beta_0\rangle$. So the three qubits are now entangled.

\begin{equation}
    \begin{split}
        |\Psi_0\rangle & = |\psi\rangle\frac{1}{\sqrt{2}}(|00\rangle + |11\rangle) \\
        |\Psi_0\rangle & = \frac{1}{\sqrt{2}}(|\psi\rangle|00\rangle + |\psi\rangle|11\rangle) \\
        |\psi\rangle & = \alpha|0\rangle + \beta|1\rangle \\
        \Longrightarrow |\Psi_0\rangle & = \frac{1}{\sqrt{2}}((\alpha|0\rangle + \beta|1\rangle)|00\rangle + (\alpha|0\rangle + \beta|1\rangle)|11\rangle) \\
        |\Psi_0\rangle & = \frac{1}{\sqrt{2}}(\alpha|000\rangle + \beta|100\rangle + \alpha|011\rangle + \beta|111\rangle) \\
        \Longrightarrow |\Psi_1\rangle & = \frac{1}{\sqrt{2}}(\alpha CNOT(|00\rangle)|\rangle0\rangle
            + \beta CNOT(|10\rangle)|0\rangle
            + \alpha CNOT(|01\rangle)|1\rangle
            + \beta CNOT(|11\rangle)|1\rangle) \\
        |\Psi_1\rangle & = \frac{1}{\sqrt{2}}
            (\alpha|000\rangle + \beta|110\rangle + \alpha|011\rangle + \beta|101\rangle) \\
    \end{split}
\end{equation}

We are applying an Hadamard gate on the first qbit.

\begin{equation}
    \begin{split}
        |\Psi_2\rangle & = \frac{1}{\sqrt{2}}
            (\alpha H(|0\rangle)|00\rangle 
            + \beta H(|1\rangle)|10\rangle 
            + \alpha H(|0\rangle)|11\rangle 
            + \beta H(|1\rangle)|01\rangle) \\
        |\Psi_2\rangle & = \frac{1}{\sqrt{2}}
            (\alpha\frac{1}{\sqrt{2}}(|0\rangle + |1\rangle)|00\rangle
            + \beta\frac{1}{\sqrt{2}}(|0\rangle - |1\rangle)|10\rangle 
            + \alpha\frac{1}{\sqrt{2}}(|0\rangle + |1\rangle)|11\rangle 
            + \beta\frac{1}{\sqrt{2}}(|0\rangle - |1\rangle)|01\rangle) \\
        |\Psi_2\rangle & = \frac{1}{2}(\alpha(|000\rangle + |100\rangle)
            + \beta(|010\rangle - |110\rangle)
            + \alpha(|011\rangle + |111\rangle)
            + \beta(|001\rangle - |101\rangle)) \\
        |\Psi_2\rangle & = \frac{1}{2}(\alpha|000\rangle + \alpha|100\rangle
            + \beta|010\rangle - \beta|110\rangle
            + \alpha|011\rangle + \alpha|111\rangle
            + \beta|001\rangle - \beta|101\rangle) \\
    \end{split}
\end{equation}

Let's do the mesurements of the first two qbits

\begin{equation}
    \begin{split}
        |\Psi_2\rangle & = \frac{1}{2}(\alpha|000\rangle + \alpha|100\rangle
            + \beta|010\rangle - \beta|110\rangle
            + \alpha|011\rangle + \alpha|111\rangle
            + \beta|001\rangle - \beta|101\rangle) \\
        |\Psi_2\rangle & = \frac{1}{2}(|00\rangle(\alpha|0\rangle + \beta|1\rangle)
            + |10\rangle(\alpha|0\rangle - \beta|1\rangle)
            + |01\rangle(\beta|0\rangle + \alpha|1\rangle)
            + |11\rangle(\beta|0\rangle - \alpha|1\rangle)) \\
    \end{split}
\end{equation}

So the two first bit mesurements are either $|00\rangle$, $|01\rangle$, $|10\rangle$, $|11\rangle$
and each of them has a probability of $\frac{1}{4}$.

We can determine the composition of the last bits depending on the mesurements of the first two bits.
\begin{tabular}{|c|c|}
    \hline
        $|00\rangle$ & $\alpha|0\rangle + \beta|1\rangle$ \\
        $|01\rangle$ & $\alpha|0\rangle - \beta|1\rangle$ \\
        $|10\rangle$ & $\beta|0\rangle + \alpha|1\rangle$ \\
        $|11\rangle$ & $\beta|0\rangle - \alpha|1\rangle$ \\
    \hline
\end{tabular}

We want the final qbit $|\Phi\rangle$ to be equal to $\alpha|0\rangle + \beta|1\rangle$
So we need to apply specific gates depending of the mesurements of the first two bits.

If the first bit of the mesurement is $|1\rangle$ You must apply the door $Z$ on the last qbit.
Because the coefficient $\beta$ of the qbit $|0\rangle$ is negative, the door $Z$ will invert the sign of the qbit.

If the second bit of the mesurement is $|1\rangle$ You must apply the door $X$ on the last qbit.
This door will invert the $\alpha$ and $\beta$ coefficients.

Finally we can find the final qbit to be equal to 
\begin{equation}
    \Longrightarrow |\psi_4\rangle = \alpha|0\rangle + \beta|1\rangle
\end{equation}

We teleported the value of the qbit
\end{document}
