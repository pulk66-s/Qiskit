\documentclass{article}
\begin{document}
Certainly! A normalized vector is obtained by scaling a non-zero vector by the reciprocal of its norm. The process of normalizing a vector produces a unit vector, which is a vector of length 1. Here’s a step-by-step proof of why the norm of a normalized vector is equal to 1:

	1.	Definition of a Normalized Vector:
Let$\mathbf{v}$ be a non-zero vector, and$|\mathbf{v}|$ be its norm. A normalized vector$\mathbf{u}$ is obtained by dividing$\mathbf{v}$ by its norm:
$$\mathbf{u} = \frac{\mathbf{v}}{|\mathbf{v}|}$$
	2.	Calculating the Norm of the Normalized Vector:
Now, we’ll calculate the norm of the normalized vector$\mathbf{u}$:
$$|\mathbf{u}| = \left|\frac{\mathbf{v}}{|\mathbf{v}|}\right|$$
	3.	Properties of Norms:
The norm has the property that$|c\mathbf{a}| = |c| \cdot |\mathbf{a}|$ for any scalar$c$ and vector$\mathbf{a}$. Applying this property, we get:
$$|\mathbf{u}| = \left|\frac{1}{|\mathbf{v}|}\right| \cdot |\mathbf{v}|$$
	4.	Simplification:
Now,$|\mathbf{v}|$ is a positive quantity (since$\mathbf{v}$ is a non-zero vector), so we can simplify further:
$$|\mathbf{u}| = \frac{1}{|\mathbf{v}|} \cdot |\mathbf{v}| = 1$$

So, we’ve shown that the norm of a normalized vector is equal to 1.
\end{document}
